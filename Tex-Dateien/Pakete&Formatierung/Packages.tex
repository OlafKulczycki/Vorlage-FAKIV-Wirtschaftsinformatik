\usepackage[ngerman]{babel}% Verwendete Sprache
\usepackage[T1]{fontenc}
\usepackage[utf8]{inputenc}
\usepackage[backend=biber,style=numeric]{biblatex} % bib und Literaturverzeichnisse
\usepackage[hang,bottom]{footmisc} % Optionen für Fußnoten
\usepackage{helvet} % Arial als Schriftart
\usepackage[a4paper, left=3cm, right=2.5cm, top=3cm, bottom=2cm]{geometry} % Seitenränder setzen
\usepackage{setspace} % Zeilenabstand setzen
\usepackage[document]{ragged2e} % Blocksatz einstellen
\usepackage{fancyhdr} % Kopf- und Fußzeilen
\usepackage{graphicx} % Grafiken einbinden
\usepackage[autostyle=true,german=quotes]{csquotes} % Anführungszeichen / Zitate
\usepackage{titling}% Bessere Verwendung von Meta-Informationen
\usepackage{titlesec}
\usepackage{comment}% Packet für das auskommentieren größerer Texte
					% \begin{comment}
					% Dein ausgekommentierter Text
					% \end{comment}
\usepackage{url} % Kann Links verwenden

%Einstellung um das Inhaktsverzeichnis anklickbar zu machen.
\usepackage[colorlinks,
pdfpagelabels,
pdfstartview = FitH,
bookmarksopen = true,
bookmarksnumbered = true,
linkcolor = black,
plainpages = false,
hypertexnames = false,
citecolor = black] {hyperref}

\usepackage{xcolor}% Ermöglicht es die Farbe des textes zu ändern, Beispiel \textcolor{blue}{Text}
\usepackage{amsmath}% Für equotion
\usepackage{float} % wird benötigt für den [H] Command bei Figures
\usepackage{acronym} % Abkürzungen und Abkürzungsverzeichniss